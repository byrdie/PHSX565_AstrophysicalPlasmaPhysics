\documentclass[10pt,letterpaper]{article}
\usepackage[margin=1in]{geometry}
\usepackage[latin1]{inputenc}
\usepackage{amsmath}
\usepackage{amsfonts}
\usepackage{amssymb}
\usepackage{graphicx}
\author{Roy Smart}
\begin{document}
	
	\section{Referee \# 1}
	
		The author of this paper lays out the process of solving parabolic and hyperbolic versions of heat conduction very clearly. They do not stray from the question of whether the explicit finite difference method will still give an increased efficiency in solving the hyperbolic heat conduction equation. With minor revisions this paper would be considered acceptable. I will address all revisions sequentially and attempt to be as clear as possible as to where these revisions are to be made. They are as follows
	
		\begin{itemize}
			\item First and foremost, the paper is lacking an abstract! Adding this is the greatest concern and will allow for readers to head into the paper with an idea of the concepts covered.
			\item Within the introduction an abbreviation for the Rempel 2017 paper could be introduced, as this is referenced several times within the body of the paper.
			\item The end of the introduction section could benefit from a brief paragraph outlining the structure of the paper.
			\item After Eq. 1 it would be useful to change the $\partial_y =\partial/\partial y$  to $\partial_x = \partial / \partial x$ to add clarity.
			\item At the end of section 2.1, simply state ``We will not consider elliptic PDEs in this work''
			\item At Eq. 7/Eq. 8 the motivation for adding the first order term in q which makes the system hyperbolic is not clear. Is this motivated physically or is it simply to make the equation hyperbolic? As you cite Rempel 2017 as your motivation, it would be enlightening to mention their reasoning as well.
			\item You could eliminate Eq. 9 and Eq. 10 and simply quote your result as you explain the mathematical reasoning in the above paragraph. Or you could remove the written explanation in favor of the mathematical one.
			\item After Eq. 14 you mention that you are scaling the equations in an interesting way. Please go into more detail about scaling your equations. This may warrant a whole paragraph or short subsection, as scaling equations is at the heart of finding equations numerically.
			\item You can remove Eq. 22 to Eq. 24 (possibly Eq. 25 as well), these intermediate steps are not necessary for the reader to reach the conclusion.
			\item Just before beginning section 4.1.1 you mention that you are evaluating temperature on the face of each grid cell and heat flux at the edge of each grid cell using a staggered grid. It may be useful to mention the physical significance of this and any computational consequences of implementing a staggered grid.
			\item When defining the first order difference method in Eq. 27 and Eq. 28 you use a forward spatial difference for the temperature differential and a backward spatial difference for the heat flux differential. Mention why the finite difference method is not the same for these.
			\item At Eq. 56 could simply state the result $|G|^2=4 \kappa \Delta t/(\Delta x^2 ) \gamma^2$, the intermediate step does not need to be addressed.
			\item Eq. 67 and Eq. 68 could be removed as well, since the intermediate steps are relatively simple.
			\item There is a misspelling of the word ?spatial? in the second paragraph of section 5
			\item It could be useful to state what value of $\kappa$ was taken for each figure shown. While it is mentioned at the beginning of the section this would help the figures stand alone.
			\item For the figures a legend could be added to indicate which specific time steps were used to create the plots.
		\end{itemize}
		
		Overall this paper offers clear and succinct treatment on how to develop an explicit finite difference method for solving PDEs and whether combining this solving method with the hyperbolic equation results in efficiency gains. The question being asked is never far from mind and all the background information is relevant. In addition to the specified revisions mentioned above a more explicit conclusion could allow the reader to make clearer connections on the structure of the paper and the results reached. This seems important as the abstract, introduction, conclusion and figures often need to act as standalone pieces of the paper which readers use to gauge how relevant the body of work is to their own research. As stated at the beginning of this referee this paper was an enjoyable and enlightening study and would be acceptable for publication with minor revisions.
	
	\section{Referee \# 2}
	
	This was a paper with a modest but interesting objective.  It is clearly written and manages to achieve its objective for the most part.
	
	\begin{itemize}
		\item The sentence following eq. (2) states that hyperbolic PDEs ``have properties we have come to expect from the physical world, namely finite information travel time.''  This seems too expansive.  Physics is replete with parabolic equations, like the traditional heat equation.
		\item The results are presented in Figs. 1 and 2, but the text does not explain what these show. Do the figures show solutions of the parabolic or hyperbolic equations?  or both?  What value of $f$ has been used?  which times steps are used?  The two equations are different, so it seems unlikely they yield the same solution.  How do the solutions differ?
	\end{itemize}
	
\end{document}