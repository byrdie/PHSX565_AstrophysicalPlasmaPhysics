\documentclass[iop]{emulateapj}
\usepackage[latin1]{inputenc}
\usepackage{amsmath}
\usepackage{amsfonts}
\usepackage{amssymb}
\usepackage{graphicx}
\usepackage{tikz}
\usepackage{float}

\usepackage{hyperref}
\usepackage[nolist]{acronym}


\begin{document}
	
	\title{Hyperbolic Heat Conduction Using First-Order Explict Finite Difference}
	\author{Roy Smart}
	\author{Dana Longcope}
	\affil{Department of Physics, Montana State University, Bozeman MT, 59717, USA}
	
	\begin{abstract}
		
	\end{abstract}
	
	\section{Introduction}
	
		\noindent
		An accurate and efficient model of heat conduction is important for models of the solar atmosphere.
		
		The most common model of heat conduction uses diffusion describe the spread of heat throughout a fluid.
			The process of diffusion is parabolic, i.e. information can instantly be communicated across the domain of a heat conducting fluid.
			This property can be prohibitive for numerical solutions of heat-conducting fluids, since it necessitates the use of extremely short timesteps.
			
		To address this problem, a hyperbolic form of heat conduction has been proposed.
			A hyperbolic formulation means that there is an upper limit on the speed of information.
				This method would be arguably more accurate than the parabolic description since heat obviously doesn't travel faster than the speed of light, for example.
			The hyperbolic form of the heat equation should allow for faster timesteps.
			
		In this study, we will investigate the performance of hyperbolic heat conduction compared to parabolic heat conduction in one dimension.
	
	\section{Description}
		\subsection{Parabolic Heat Conduction}
		
			Parabolic heat conduction is often known simply as the heat equation and is given in one dimension as
			\begin{equation} \label{par}
				\frac{\partial T}{\partial t} - \kappa \frac{\partial^2 T}{\partial x^2} = 0
			\end{equation}
			where $T$ is the temperature, $x$ is a spatial variable, $t$ is time, and $\kappa$ is the thermal diffusivity.

			The diffusion operator in the second term causes some initial condition to be blurred out.
		
		\subsection{Hyperbolic Heat Conduction}
		
			Hyperbolic heat conduction equations have been proposed by a number of researchers in astrophysics and solar physics.
				Recently, \cite{A} used a hyperbolic form of the heat conduction equation in the MURaM radiative \ac{MHD} code to form a comprehensive simulation of the solar atmosphere.
				To illustrate their approach, \cite{A} presented the following one-dimensional system of equations.
				\begin{gather}
					\tau \frac{\partial q}{\partial t} + q = -\kappa \frac{\partial T}{\partial x} \label{flux}\\
					\frac{\partial T}{\partial t} = -\frac{\partial q}{\partial x} \label{temp}
				\end{gather}
				where $q$ is the heat flux and $\tau$ is some characteristic time scale. 
				Notice that if $\tau = 0$, the system is equivalent to the parabolic form of the heat equation, expressed as two first-order equations.
				
				If we assume that $\kappa$ and $\tau$ are constants, we can express Equation \ref{flux} as a single second-order equation by taking a spatial derivative and plugging in Equation \ref{temp}.
				\begin{align}
					& \frac{\partial}{\partial x} \left( \tau \frac{\partial q}{\partial t} \right) + \frac{\partial q}{\partial x} = - \frac{\partial}{\partial x} \left( \kappa \frac{\partial T}{\partial x} \right) \\
					\Rightarrow \quad & \tau \frac{\partial}{\partial t} \frac{\partial q}{\partial x} + \frac{\partial q}{\partial x} = -\kappa \frac{\partial^2 T}{\partial x^2} \\
					\Rightarrow \quad & \tau \frac{\partial^2 T}{\partial t^2} +  \frac{\partial T}{\partial t} - \kappa \frac{\partial^2 T}{\partial x^2} = 0 \label{wave}
				\end{align}
				The above equation is a linear, damped wave equation in temperature. 
				
				While the solar atmosphere has a temperature-dependent thermal diffusivity $\kappa$, Equation \ref{wave} will still be useful for examining the linear behavior of the system formed by Equations \ref{flux} and \ref{temp}.
		
	\section{Linear Analysis}
	
			A linear analysis of these differential equations will be helpful towards understanding their behavior.
	
		\subsection{Dispersion Relation}
		
				We can derive a \ac{DR} for each heat conduction case.
					These will be used to understand the range of allowable values for $\tau$.
					The \ac{DR}s can also be useful for estimating the maximum timestep size required to maintain stability.
					
			\subsubsection{Parabolic}
			
				Deriving a \ac{DR} for the parabolic case is much simpler, so we'll start with that.
					As in the previous section, $\kappa$ will be taken to be a constant, so we can carry out a linear analysis.
					
				Taking the Fourier transform of Equation \ref{par} yields
				\begin{equation}
					i \omega \tilde{T} + \kappa k^2 \tilde{T} = 0,
				\end{equation}
				where $\omega$ is the temporal frequency, $k$ is the spatial wavenumber, $\tilde{T}$ is the Fourier transform of the temperature, and $i$ is the imaginary unit.
				Canceling $\tilde{T}$ from both terms and solving for the frequency gives the parabolic \ac{DR}
				\begin{equation}
					\omega(k) = i \kappa k^2.
				\end{equation}
				From the above \ac{DR} we can see that $\omega$ is imaginary. Thus, all wavenumbers decay exponentially, which is the expected behavior from the diffusion operator.
						
			\subsubsection{Hyperbolic}
			
				Similarly, for the hyperbolic case, we take the Fourier transform of Equation \ref{wave} and cancel the factors of $\tilde{T}$
				\begin{equation}
					- \tau \omega^2 + i \omega + \kappa k^2 = 0.
				\end{equation}
				Solve for $\omega$ to find the \ac{DR} using the quadratic equation
				\begin{equation}
					\Rightarrow \; \omega(k) = \frac{1}{2 \tau} \left( i \pm \sqrt{4 \kappa \tau k^2 - 1} \right).
				\end{equation}
				The first term is always imaginary, leading to decaying modes in the solution.
				The second term is real and thus oscillatory if 
				\begin{equation} \label{re_ineq}
					 4 \kappa \tau k^2 > 1.
				\end{equation}
				Inequality \ref{re_ineq} can be expressed as a condition on $\tau$
				\begin{equation} \label{tau_o}
					\tau > \frac{1}{4 \kappa k^2}.
				\end{equation}			
				If Inequality \ref{tau_o} is satisfied, the magnitude of $\omega$ is
				\begin{align}
					|\omega(k)| &= \frac{1}{2 \tau} \sqrt{ 1^2 +  \left(\pm \sqrt{4 \kappa \tau k^2 - 1}\right)^2 } \\
								&= \frac{1}{2 \tau} \sqrt{1 + (4 \kappa \tau k^2 - 1)} \\
								&= \frac{1}{2 \tau} \sqrt{4 \kappa \tau k^2} \\
								&= \sqrt{\frac{4 \kappa \tau k^2}{4 \tau^2}} \\
								&= \sqrt{\frac{\kappa}{\tau}} k \\
								&= c k
				\end{align}
				where $c = \sqrt{\kappa / \tau}$ is some characteristic speed.
				
				If Inequality \ref{tau_o} is not satisfied, the magnitude of $\omega$ is then
				\begin{align}
					|\omega(k)| &= \frac{1}{2 \tau} \sqrt{\left( 1 \pm \sqrt{1 - 4 \kappa \tau k^2} \right)^2} \\
								&= \frac{1}{2 \tau} \left[ 1 \pm \sqrt{1 - 4 \kappa \tau k^2} \; \right]
				\end{align}
				Since $4 \kappa \tau k^2 < 1$, we can perform a binomial expansion on the denominator
				\begin{align}
					|\omega(k)| &\approx \frac{1}{2 \tau} \left[ 1 \pm \left( 1 - 2 \kappa \tau k^2 \right) \right] \\
								&= \frac{1}{\tau} - \kappa k^2, \quad \kappa k^2
				\end{align}

			
		\subsection{Analytic Solution}
			\subsubsection{Parabolic}
			\subsubsection{Hyperbolic}
	\section{Numerical Implementation}
	
		We would like to incorporate this hyperbolic heat conduction as part of a larger simulations of the solar atmosphere.
			In the case of the solar atmosphere, $\kappa$ is in general not constant, resulting in non-linear \acp{PDE}.
			Therefore, we must resort to numerical methods to solve the hyperbolic heat conduction equations in the environment of the solar atmosphere.
	
		\subsection{Explicit Finite-difference}
		
			One of the simplest and most popular numerical methods for solving non-linear \acp{PDE} is known as \ac{FD}.
				The \ac{FD} method works by approximating some function using a truncated power series, and then taking the derivative of the power series to approximate the derivative of the function.
				Here, we will be using the simplest form of \ac{FD}, first-order \ac{FD}, where derivatives are calculated using linear interpolation.
				
			In this work, we will be using a form of \ac{FD} known as \ac{EFD}. 
				Under \ac{EFD}, the state of some system at some time is calculated using only the state of the system at a previous time. 
				This is in contrast to \ac{IFD}, where the state of the system at the current time is found by solving an equation involving not only the state of the system at a previous time, but at the current time as well.
				\ac{IFD} has the benefit of being more stable, but is ill-suited for massively parallel processors, such as supercomputer clusters and \acp{GPU}.
				Because utilizing massively parallel processors is necessary to remain competitive, we will be restricting this analysis to \ac{EFD}.
				
				In the follwing sections, we will derive update rules for both the parabolic and hyperbolic forms of heat conduction.
					Update rules are recursion relations that explain how to update the values of the dependent variables for one timestep.
					
				For both the parabolic and hyperbolic cases, we will express the dependent variables, $q$ and $T$, on a staggered grid.
					Temperature will be evaluated in the center of a grid cell, while heat flux will be evaluated on the edges of a grid cell.
		
			\subsubsection{Parabolic}
			
				
			
				When using \ac{EFD}, it is often helpful to express all second-order equations as systems of first-order equations.
					Noticing that Equation \ref{par} is indeed second order, we will instead start with the system formed by Equations \ref{flux} and \ref{temp} with $\tau$ set to zero.
				\begin{gather}
					q = -\kappa \frac{\partial T}{\partial x} \\
					\frac{\partial T}{\partial t} = -\frac{\partial q}{\partial x} 
				\end{gather}
				Discretize the above system of differential equations by taking $n$ to be the index in time, $i$ to be the index in space, $\Delta t$ to be the timestep size, and $\Delta x$ to be the spatial grid size.
				\begin{gather}
					q_i^{n+1} = - \kappa \frac{T_{i+1}^n - T_i^n}{\Delta x} \\
					\frac{T_i^{n+1} - T_i^n}{\Delta t} = - \frac{q_i^n - q_{i-1}^n}{\Delta x}
				\end{gather}
				Solve for $q_i^{n+1}$ and $T_i^{n+1}$ to express the system as an update rule
				\begin{align}
						q_i^{n+1} &= - \kappa \frac{T_{i+1}^n - T_i^n}{\Delta x} \label{q_par_up}\\
						T_i^{n+1} &= T_i^n - \frac{1}{v} \left( q_i^n - q_{i-1}^n \right), \label{T_par_up}
				\end{align}
				where $v = \Delta x / \Delta t$, is some characteristic velocity for the discrete grid.
				
				Equations \ref{q_par_up} and \ref{T_par_up} are the update rules for the parabolic form of heat conduction. 
					Given initial conditions on temperature and boundary conditions on either temperature or heat flux is sufficient information to solve the \ac{PDE}.
					
			\subsubsection{Hyperbolic}
			
				For the hyperbolic case, we will discretize Equations \ref{flux} and \ref{temp}
				\begin{gather}
					\tau \frac{q_i^{n+1} - q_i^n}{\Delta t} + q_i^n =  \kappa \frac{T_{i+1}^n - T_i^n}{\Delta x} \label{q_hyp} \\
					\frac{T_i^{n+1} - T_i^n}{\Delta t} = - \frac{q_i^n - q_{i-1}^n}{\Delta x} \label{T_hyp}
				\end{gather}
				Notice that the second term on the \ac{LHS} is evaluated at $n$ instead of $n+1$ as in the parabolic case.
					This is an arbitrary choice and could work either way.
					
				Like the parabolic case, we will express Equations \ref{T_hyp} and \ref{q_hyp} as an update rule by solving for $q_i^{n+1}$ and $T_i^{n+1}$.
				\begin{align}
					q_i^{n+1} &= R q_i^n - \frac{c^2}{v} \left( T_{i+1}^n - T_i^n \right) \\
					T_i^{n+1} &= T_i^n - \frac{1}{v} \left( q_i^n - q_{i-1}^n \right),
				\end{align}
				where $R = (\tau - \Delta t) / \tau$
			
		\subsection{Von Neumann Stability Analysis}
			\subsubsection{Parabolic}
			\subsubsection{Hyperbolic}
	\section{Results}
	\section{Conclusion}
	\section{Appendix}
			
	\bibliographystyle{apj}
	\bibliography{sources}
	
	\begin{acronym}
		\acro{MHD}{magnetohydrodynamics}
		\acro{DR}{dispersion relation}
		\acro{PDE}{partial differential equation}
		\acro{FD}{finite-difference}
		\acro{EFD}{explicit finite-difference}
		\acro{IFD}{implicit finite-difference}
		\acro{GPU}{graphics processing unit}
		\acro{LHS}{left-hand side}
		\acro{RHS}{right-hand side}
	\end{acronym}
	
\end{document}