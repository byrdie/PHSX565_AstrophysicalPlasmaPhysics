\documentclass[10pt,letterpaper]{article}
\usepackage[latin1]{inputenc}
\usepackage[margin=1in]{geometry}
\usepackage{amsmath}
\usepackage{amsfonts}
\usepackage{amssymb}
\usepackage{graphicx}
\title{PHSX 567 \\ Astrophysical Plasma Physics \\ Final Project Proposal}
\author{Roy Smart}
\begin{document}
	
	\maketitle
	
	\noindent
	The solar atmosphere is a chaotic environment characterized by million-degree plasma driven by intense magnetic fields originating deep within the solar convective zone. 
	The statistical behavior of this plasma is governed by the equations of magnetohydrodynamics (MHD) which use conservation of mass, momentum and energy, along with Maxwell's equations to predict future configurations of the plasma given some initial condition. 
	Unfortunately, the MHD equations have no known analytic solutions over an arbitrary domain, so solutions must be approximated using numerical techniques.
	
	Possibly the most popular technique for numerical solutions of the MHD equations is known as explicit finite-difference, where
	
\end{document}