%% AMS-LaTeX Created with the Wolfram Language for Students - Personal Use Only : www.wolfram.com

\documentclass{article}
\usepackage{amsmath, amssymb, graphics, setspace}

\newcommand{\mathsym}[1]{{}}
\newcommand{\unicode}[1]{{}}

\newcounter{mathematicapage}
\begin{document}

PHSX 565 Astrophysical Plasma Physics\\
Problem Set 4 - Accretion Disks\\
Roy Smart

\subsection*{Prepare Mathematica environment}

\subsubsection*{Clear variables}

\begin{doublespace}
\noindent\(\pmb{\text{Clear}[\text{{``}Global$\grave{ }$*{''}}]}\\
\pmb{\text{vals} = \{\};}\)
\end{doublespace}

\subsubsection*{Define shortcut to convert rule to equation}

\begin{doublespace}
\noindent\(\pmb{\text{r2e} = \text{Rule} \to  \text{Equal};}\)
\end{doublespace}

\subsubsection*{Use preprint variable to print derivatives in traditional form}

\begin{doublespace}
\noindent\(\pmb{\text{$\$$PrePrint}=\#\text{/.}\text{Derivative}[\text{id$\_\_$}][\text{f$\_$}][\text{args$\_\_$}]:\to \text{TraditionalForm}[\text{HoldForm}@D[f[\text{args}],\#]\&[\text{Sequence}\text{@@}(\text{DeleteCases}[\text{Transpose}[\{\{\text{args}\},\{\text{id}\}\}],\{\_,0\}]\text{/.}\{\text{x$\_$},1\}:\to
x)]]\&;}\)
\end{doublespace}

\subsubsection*{Define the del operator}

\begin{doublespace}
\noindent\(\pmb{\text{$\$$x} = \{\varpi ,\phi ,z\}}\\
\pmb{\triangledown \text{/:}\triangledown  \text{f$\_$}\text{:=}\text{Grad}[f,\text{$\$$x},\text{{``}Cylindrical{''}}]}\\
\pmb{\triangledown \cdot \text{f$\_$}\text{:=}\text{Div}[f,\text{$\$$x},\text{{``}Cylindrical{''}}]}\\
\pmb{\triangledown \text{/:}\triangledown \times \text{f$\_$}\text{:=}\text{Curl}[f,\text{$\$$x},\text{{``}Cylindrical{''}}]}\\
\pmb{\triangledown \cdot \triangledown \text{:=}\Delta }\\
\pmb{\triangledown \text{/:}\triangledown {}^{\wedge}2\text{:=}\Delta }\\
\pmb{\Delta \text{/:}\Delta  \text{f$\_$}\text{:=}\text{Laplacian}[f,\text{$\$$x},\text{{``}Cylindrical{''}}]}\\
\pmb{\text{CenterDot}\text{/:}(\text{v$\_$}\cdot \triangledown ) \text{f$\_$}\text{:=}\text{Grad}[f,\text{$\$$x},\text{{``}Cylindrical{''}}].v}\)
\end{doublespace}

\begin{doublespace}
\noindent\(\{\varpi ,\phi ,z\}\)
\end{doublespace}

\subsection*{Save the information provided in the problem statement}

\subsubsection*{The average mass of the plasma particle is}

\begin{doublespace}
\noindent\(\pmb{\text{m$\$$} = m\to  \text{mp}/2;}\\
\pmb{\text{$\$$Assumptions} = m >0 \&\& \text{mp} > 0;}\)
\end{doublespace}

\subsubsection*{The dominant, azimuthal flow velocity is given in cylindrical coordinates as}

\begin{doublespace}
\noindent\(\pmb{\text{u$\$$0} = u[\varpi ,z]\to  \{0,\varpi  \Omega [\varpi ],0\};}\\
\pmb{\text{$\$$Assumptions} = \text{$\$$Assumptions} \&\& \varpi  > 0;}\)
\end{doublespace}

\subsubsection*{The gravitational potential is approximated as}

\begin{doublespace}
\noindent\(\pmb{\text{$\psi \$$0} = \psi [\varpi ] \to  -\frac{G M}{\varpi }+\frac{1}{2}\text{$\Omega $k}[\varpi ]^2z^2;}\\
\pmb{\text{$\$$Assumptions} = \text{$\$$Assumptions} \&\& G > 0 \&\& M >0 \&\& z \in  \text{Reals} \&\& \text{$\Omega $k}[\varpi ] > 0;}\\
\pmb{\text{AppendTo}[\text{vals}, G\to  \text{UnitConvert}[\text{Quantity}[\text{{``}GravitationalConstant{''}}]] ];}\)
\end{doublespace}

\subsubsection*{Where the Keplarian rotation is}

\begin{doublespace}
\noindent\(\pmb{\text{$\Omega $k$\$$0} = \text{$\Omega $k}[\varpi ]\to \sqrt{\frac{G M}{\varpi ^3}};}\)
\end{doublespace}

\section*{Part a.}

\subsection*{Write the axial component of the momentum equation using the azimuthal flow velocity and the gravitational potential}

\subsubsection*{The momentum equation in a gravitational potential is given as}

\begin{doublespace}
\noindent\(\pmb{\text{M$\$$}[\rho \_,\text{u$\_$},\text{p$\_$},\psi \_] \text{:=} \rho \left(\partial _tu\right) + \rho  \left(\left(\frac{1}{2}(\triangledown
(u.u))\right)-(u\times (\triangledown \times u))\right)== -(\triangledown  p)-\rho  (\triangledown  \psi );}\)
\end{doublespace}

\subsubsection*{Evaluate the z-component of the momentum equation using the information given in the problem statement}

\begin{doublespace}
\noindent\(\pmb{(\text{Mz$\$$a} = \text{M$\$$}[\rho [\varpi ,z],u[\varpi ,z]\text{/.}\text{u$\$$0} ,p[\varpi ,z],\psi [\varpi ]\text{/.}\text{$\psi
\$$0}\text{/.}\text{$\Omega $k$\$$0}][[\text{;;},3]]\text{   }\text{//}\text{Simplify}) \text{//}\text{Framed}}\)
\end{doublespace}

\begin{doublespace}
\noindent\(\fbox{$\varpi ^3 \frac{\partial p(\varpi ,z)}{\partial z}+G M z \rho [\varpi ,z]==0$}\)
\end{doublespace}

\subsection*{Find the pressure within the disk}

\subsubsection*{Use the ideal gas law}

\begin{doublespace}
\noindent\(\pmb{\text{igl$\$$} = p[\varpi ,z] \text{==} \frac{\text{kB}}{m}\rho [\varpi ,z]T[\varpi ];}\\
\pmb{\text{$\$$Assumptions} = \text{$\$$Assumptions} \&\& \text{kB} > 0;}\\
\pmb{\text{AppendTo}[\text{vals}, \text{kB}\to  (\text{UnitConvert}[\text{Quantity}[\text{{``}Boltzmann Constant{''}}]]\text{/.}\text{{``}Kelvins{''}}\to
\text{{``}KelvinsDifference{''}})];}\\
\pmb{\text{AppendTo}[\text{vals},m\to  \text{UnitConvert}[\text{Quantity}[\text{{``}ProtonMass{''}}]]]}\)
\end{doublespace}

\begin{doublespace}
\noindent\(\left\{G\to ,\text{kB}\to ,m\to \right\}\)
\end{doublespace}

\subsubsection*{To write density in terms of pressure and temperature}

\begin{doublespace}
\noindent\(\pmb{\text{$\rho \$$a} = \text{Solve}[\text{igl$\$$}, \rho [\varpi ,z]][[1,1]]}\)
\end{doublespace}

\begin{doublespace}
\noindent\(\rho [\varpi ,z]\to \frac{m p[\varpi ,z]}{\text{kB} T[\varpi ]}\)
\end{doublespace}

\subsubsection*{and solve the differential equation found in the first part of this problem}

\begin{doublespace}
\noindent\(\pmb{\text{p$\$$a} = \text{Mz$\$$a}}\\
\pmb{\text{p$\$$a}= \text{p$\$$a}\text{/.} \text{$\rho \$$a}}\\
\pmb{(\text{p$\$$a} = \text{DSolve}[\{\text{p$\$$a} , p[\varpi ,0] == \text{p0}[\varpi ]\},p[\varpi ,z],z][[1,1]]) \text{//}\text{Framed}}\)
\end{doublespace}

\begin{doublespace}
\noindent\(\varpi ^3 \frac{\partial p(\varpi ,z)}{\partial z}+G M z \rho [\varpi ,z]==0\)
\end{doublespace}

\begin{doublespace}
\noindent\(\frac{G m M z p[\varpi ,z]}{\text{kB} T[\varpi ]}+\varpi ^3 \frac{\partial p(\varpi ,z)}{\partial z}==0\)
\end{doublespace}

\begin{doublespace}
\noindent\(\fbox{$p[\varpi ,z]\to e^{-\frac{G m M z^2}{2 \text{kB} \varpi ^3 T[\varpi ]}} \text{p0}[\varpi ]$}\)
\end{doublespace}

\subsection*{Show that the vertical scale height is the ratio of the isothermal sound speed to the Keplarian rotation rate.}

\subsubsection*{The isothermal sound speed is the usual}

\begin{doublespace}
\noindent\(\pmb{\text{csi$\$$a} = \text{csi}[\varpi ] \to \text{  }\sqrt{\text{kB} T[\varpi ]/m};}\\
\pmb{\text{$\$$Assumptions} = \text{$\$$Assumptions} \&\& T[\varpi ] > 0 \&\& \text{csi}[\varpi ] > 0;}\)
\end{doublespace}

\subsubsection*{Write the pressure found in the previous part of the problem in terms of the isothermal sound speed and the Keplarian rotation rate}

\begin{doublespace}
\noindent\(\pmb{\text{p1$\$$a} = \text{p$\$$a}\text{  }}\\
\pmb{\text{p1$\$$a} = \text{p1$\$$a}\text{/.} \text{Solve}[\text{csi$\$$a}\text{/.}\text{r2e},T[\varpi ]][[1,1]]}\\
\pmb{\text{p1$\$$a} = \text{p1$\$$a}\text{/.} \text{Solve}[\text{$\Omega $k$\$$0}\text{/.} \text{r2e}, M][[1,1]]}\)
\end{doublespace}

\begin{doublespace}
\noindent\(p[\varpi ,z]\to e^{-\frac{G m M z^2}{2 \text{kB} \varpi ^3 T[\varpi ]}} \text{p0}[\varpi ]\)
\end{doublespace}

\begin{doublespace}
\noindent\(p[\varpi ,z]\to e^{-\frac{G M z^2}{2 \varpi ^3 \text{csi}[\varpi ]^2}} \text{p0}[\varpi ]\)
\end{doublespace}

\begin{doublespace}
\noindent\(p[\varpi ,z]\to e^{-\frac{z^2 \text{$\Omega $k}[\varpi ]^2}{2 \text{csi}[\varpi ]^2}} \text{p0}[\varpi ]\)
\end{doublespace}

\subsubsection*{Now write in terms of the scale height}

\begin{doublespace}
\noindent\(\pmb{\text{h$\$$a} = h[\varpi ]\to  \frac{\text{csi}[\varpi ]}{\text{$\Omega $k}[\varpi ]};}\\
\pmb{\text{$\$$Assumptions} = \text{$\$$Assumptions} \&\& h[\varpi ] >0;}\)
\end{doublespace}

\begin{doublespace}
\noindent\(\pmb{\text{p2$\$$a} =\text{p1$\$$a}}\\
\pmb{(\text{p2$\$$a} =\text{p2$\$$a} \text{/.} \text{Solve}[\text{h$\$$a} \text{/.} \text{r2e}, \text{csi}[\varpi ]][[1,1]]) \text{//}\text{Framed}}\)
\end{doublespace}

\begin{doublespace}
\noindent\(p[\varpi ,z]\to e^{-\frac{z^2 \text{$\Omega $k}[\varpi ]^2}{2 \text{csi}[\varpi ]^2}} \text{p0}[\varpi ]\)
\end{doublespace}

\begin{doublespace}
\noindent\(\fbox{$p[\varpi ,z]\to e^{-\frac{z^2}{2 h[\varpi ]^2}} \text{p0}[\varpi ]$}\)
\end{doublespace}

\subsection*{Calculate the scale height using example parameters}

\subsubsection*{Save the given parameters to memory}

\begin{doublespace}
\noindent\(\pmb{\text{AppendTo}[\text{vals},M \to \text{  }100\text{Quantity}[\text{{``}SolarMass{''}}]];}\\
\pmb{\text{AppendTo}[\text{vals},\varpi \to  \text{Quantity}[\text{{``}AstronomicalUnit{''}}]];}\\
\pmb{\text{AppendTo}\left[\text{vals},T[\varpi ]\to  10^4\text{Quantity}[\text{{``}KelvinsDifference{''}}]\right];}\)
\end{doublespace}

\subsubsection*{Evaluate the scale height}

\begin{doublespace}
\noindent\(\pmb{\text{h$\$$a1} = \text{h$\$$a}}\\
\pmb{\text{h$\$$a1} = \text{h$\$$a1} \text{/.} \text{csi$\$$a}\text{  }\text{//}\text{Simplify}}\\
\pmb{\text{h$\$$a1} = \text{h$\$$a1} \text{/.} \text{$\Omega $k$\$$0} \text{//}\text{Simplify}}\\
\pmb{(\text{h$\$$a2} =\text{h$\$$a1} \text{/.} \text{vals} ) \text{//}\text{Framed}}\)
\end{doublespace}

\begin{doublespace}
\noindent\(h[\varpi ]\to \frac{\text{csi}[\varpi ]}{\text{$\Omega $k}[\varpi ]}\)
\end{doublespace}

\begin{doublespace}
\noindent\(h[\varpi ]\to \frac{\sqrt{\frac{\text{kB} T[\varpi ]}{m}}}{\text{$\Omega $k}[\varpi ]}\)
\end{doublespace}

\begin{doublespace}
\noindent\(h[\varpi ]\to \frac{1}{\sqrt{\frac{G m M}{\text{kB} \varpi ^3 T[\varpi ]}}}\)
\end{doublespace}

\begin{doublespace}
\noindent\(\fbox{$h[]\to $}\)
\end{doublespace}

\subsubsection*{Compare the scale height to the radial coordinate}

\begin{doublespace}
\noindent\(\pmb{\frac{h[\varpi ]}{\varpi }\to  \left(\frac{h[\varpi ]}{\varpi }\text{/.} \text{h$\$$a}\text{/.} \text{csi$\$$a}\text{/.} \text{$\Omega
$k$\$$0}\right) \text{/.} \text{vals} \text{//}\text{Framed}}\)
\end{doublespace}

\begin{doublespace}
\noindent\(\fbox{$h[] \left(\right)\to 0.03050$}\)
\end{doublespace}

\subsubsection*{So the scale height at 1 AU is only 0.03 AU. This can be considered thin.}

\section*{Part b.}

\subsection*{Write the disk{'}s surface density in terms of the scale height and the equatorial mass density}

\subsubsection*{Start by using the ideal gas law to find the density from the pressure calculated in Part a.}

\begin{doublespace}
\noindent\(\pmb{\text{$\rho \$$b} = \text{$\rho \$$a} }\\
\pmb{\text{$\rho \$$b} = \text{$\rho \$$b} \text{/.} \text{p2$\$$a} }\)
\end{doublespace}

\begin{doublespace}
\noindent\(\rho [\varpi ,z]\to \frac{m p[\varpi ,z]}{\text{kB} T[\varpi ]}\)
\end{doublespace}

\begin{doublespace}
\noindent\(\rho [\varpi ,z]\to \frac{e^{-\frac{z^2}{2 h[\varpi ]^2}} m \text{p0}[\varpi ]}{\text{kB} T[\varpi ]}\)
\end{doublespace}

\subsubsection*{Write in terms of the density at z=0}

\begin{doublespace}
\noindent\(\pmb{\text{$\rho \$$b1} = \rho [\varpi ,z]\to \text{$\rho $0}[\varpi ]e^{-\frac{z^2}{2 h[\varpi ]^2}};}\)
\end{doublespace}

\subsubsection*{Evaluate the density at z=0 in terms of the pressure at z=0}

\begin{doublespace}
\noindent\(\pmb{\text{$\rho $0$\$$b} = \text{Solve}[ (\rho [\varpi ,z] \text{/.} \text{$\rho \$$b}) ==\text{  }(\rho [\varpi ,z] \text{/.} \text{$\rho
\$$b1}), \text{$\rho $0}[\varpi ]][[1,1]]}\)
\end{doublespace}

\begin{doublespace}
\noindent\(\text{$\rho $0}[\varpi ]\to \frac{m \text{p0}[\varpi ]}{\text{kB} T[\varpi ]}\)
\end{doublespace}

\subsubsection*{Evaluate the surface density using the given formula}

\begin{doublespace}
\noindent\(\pmb{\left(\text{$\Sigma \$$b} = \Sigma [\varpi ]\to  \int _{-\infty }^{\infty }\rho [\varpi ,z] dz\text{/.}\text{$\rho \$$b1}\right)
\text{//}\text{Framed}}\)
\end{doublespace}

\begin{doublespace}
\noindent\(\fbox{$\Sigma [\varpi ]\to \sqrt{2 \pi } h[\varpi ] \text{$\rho $0}[\varpi ]$}\)
\end{doublespace}

\subsubsection*{Write the density at z=0 in terms of the surface density}

\begin{doublespace}
\noindent\(\pmb{\text{$\rho $0$\$$b1} = \text{Solve}[\text{$\Sigma \$$b} \text{/.} \text{r2e}, \text{$\rho $0}[\varpi ]][[1,1]]}\)
\end{doublespace}

\begin{doublespace}
\noindent\(\text{$\rho $0}[\varpi ]\to \frac{\Sigma [\varpi ]}{\sqrt{2 \pi } h[\varpi ]}\)
\end{doublespace}

\section*{Part c.}

\subsection*{Write down the radial component of the momentum equation evaluated in the equatorial plane}

\begin{doublespace}
\noindent\(\pmb{\text{Mz$\$$c} = \text{M$\$$}[\rho [\varpi ,z],u[\varpi ,z]\text{/.}\text{u$\$$0} ,p[\varpi ,z],\psi [\varpi ]\text{/.}\text{$\psi
\$$0}\text{/.}\text{$\Omega $k$\$$0}] [[\text{;;},1]]\text{  }}\\
\pmb{\text{Mz$\$$c} = \text{Mz$\$$c} \text{/.} \text{Solve}[\text{$\Omega $k$\$$0}\text{/.} \text{r2e}, M][[1,1]]\text{//}\text{Simplify}}\\
\pmb{(\text{Mz$\$$c} = \text{Mz$\$$c}\text{//}\text{Simplify}) \text{/.} z\to  0\text{//}\text{Framed}}\)
\end{doublespace}

\begin{doublespace}
\noindent\(\rho [\varpi ,z] \left(-\varpi ^2 \frac{\partial \Omega (\varpi )}{\partial \varpi } \Omega [\varpi ]-2 \varpi  \Omega [\varpi ]^2+\frac{1}{2}
\left(2 \varpi ^2 \frac{\partial \Omega (\varpi )}{\partial \varpi } \Omega [\varpi ]+2 \varpi  \Omega [\varpi ]^2\right)\right)==-\frac{\partial
p(\varpi ,z)}{\partial \varpi }-\left(-\frac{3 G M z^2}{2 \varpi ^4}+\frac{G M}{\varpi ^2}\right) \rho [\varpi ,z]\)
\end{doublespace}

\begin{doublespace}
\noindent\(\rho [\varpi ,z] \left(2 \varpi ^2 \Omega [\varpi ]^2+\left(3 z^2-2 \varpi ^2\right) \text{$\Omega $k}[\varpi ]^2\right)==2 \varpi  \frac{\partial
p(\varpi ,z)}{\partial \varpi }\)
\end{doublespace}

\begin{doublespace}
\noindent\(\fbox{$\rho [\varpi ,0] \left(2 \varpi ^2 \Omega [\varpi ]^2-2 \varpi ^2 \text{$\Omega $k}[\varpi ]^2\right)==2 \varpi  \frac{\partial
p(\varpi ,0)}{\partial \varpi }$}\)
\end{doublespace}

\subsection*{Use scaling arguments to show that the pressure term is negligible }

\subsubsection*{Plug the pressure found in Part a. and the density found in Part b.}

\begin{doublespace}
\noindent\(\pmb{\text{p$\$$c} = \{p :\to  \text{Function}[\{\varpi ,z\},p[\varpi ,z]\text{/.}\text{p2$\$$a}]\};}\\
\pmb{\text{p0$\$$c} = \{p :\to  \text{Function}[\{\varpi ,z\},\text{p0}[\varpi ]\text{/.}\text{p2$\$$a}]\};}\\
\pmb{\text{$\$$Assumptions} = \text{$\$$Assumptions} \&\& \text{p0}[\varpi ] > 0 ;}\\
\pmb{\text{Mz$\$$c1}=\text{Mz$\$$c}\text{/.} \text{Equal} \to  \text{Greater}}\\
\pmb{\text{Mz$\$$c1}=\text{Mz$\$$c1} \text{/.} \text{p$\$$c}\text{   }}\\
\pmb{\text{Mz$\$$c1}=\text{Mz$\$$c1} \text{/.} \text{$\rho \$$b}\text{  }}\\
\pmb{\text{Mz$\$$c1}=\text{Mz$\$$c1} \text{/.} z\to  0}\\
\pmb{\text{Mz$\$$c1}=\text{Mz$\$$c1}\text{  }\text{//}\text{Simplify}}\)
\end{doublespace}

\begin{doublespace}
\noindent\(\rho [\varpi ,z] \left(2 \varpi ^2 \Omega [\varpi ]^2+\left(3 z^2-2 \varpi ^2\right) \text{$\Omega $k}[\varpi ]^2\right)>2 \varpi  \frac{\partial
p(\varpi ,z)}{\partial \varpi }\)
\end{doublespace}

\begin{doublespace}
\noindent\(\rho [\varpi ,z] \left(2 \varpi ^2 \Omega [\varpi ]^2+\left(3 z^2-2 \varpi ^2\right) \text{$\Omega $k}[\varpi ]^2\right)>2 \varpi  \left(\frac{e^{-\frac{z^2}{2
h[\varpi ]^2}} z^2 \text{p0}[\varpi ] \frac{\partial h(\varpi )}{\partial \varpi }}{h[\varpi ]^3}+e^{-\frac{z^2}{2 h[\varpi ]^2}} \frac{\partial
\text{p0}(\varpi )}{\partial \varpi }\right)\)
\end{doublespace}

\begin{doublespace}
\noindent\(\frac{e^{-\frac{z^2}{2 h[\varpi ]^2}} m \text{p0}[\varpi ] \left(2 \varpi ^2 \Omega [\varpi ]^2+\left(3 z^2-2 \varpi ^2\right) \text{$\Omega
$k}[\varpi ]^2\right)}{\text{kB} T[\varpi ]}>2 \varpi  \left(\frac{e^{-\frac{z^2}{2 h[\varpi ]^2}} z^2 \text{p0}[\varpi ] \frac{\partial h(\varpi
)}{\partial \varpi }}{h[\varpi ]^3}+e^{-\frac{z^2}{2 h[\varpi ]^2}} \frac{\partial \text{p0}(\varpi )}{\partial \varpi }\right)\)
\end{doublespace}

\begin{doublespace}
\noindent\(\frac{m \text{p0}[\varpi ] \left(2 \varpi ^2 \Omega [\varpi ]^2-2 \varpi ^2 \text{$\Omega $k}[\varpi ]^2\right)}{\text{kB} T[\varpi ]}>2
\varpi  \frac{\partial \text{p0}(\varpi )}{\partial \varpi }\)
\end{doublespace}

\begin{doublespace}
\noindent\(m \varpi  \text{p0}[\varpi ] \left(\Omega [\varpi ]^2-\text{$\Omega $k}[\varpi ]^2\right)>\text{kB} T[\varpi ] \frac{\partial \text{p0}(\varpi
)}{\partial \varpi }\)
\end{doublespace}

\subsubsection*{Replace derivatives with ratios of the variables}

\begin{doublespace}
\noindent\(\pmb{\text{Mz$\$$c2} = \text{Mz$\$$c1} \text{/.} D[\text{p0}[\varpi ],\varpi ]\to  \text{p0}[\varpi ]/\varpi }\\
\pmb{\text{Mz$\$$c2} = \text{Mz$\$$c2} \text{//}\text{Simplify}}\\
\pmb{\text{Mz$\$$c2} = \text{Mz$\$$c2} \text{/.} \text{Solve}[\text{h$\$$a1} \text{/.} \text{r2e}, T[\varpi ]] [[1]]}\\
\pmb{(\text{Mz$\$$c2} = \text{Mz$\$$c2}\text{  }\text{//}\text{Simplify}) \text{//}\text{Framed}}\)
\end{doublespace}

\begin{doublespace}
\noindent\(m \varpi  \text{p0}[\varpi ] \left(\Omega [\varpi ]^2-\text{$\Omega $k}[\varpi ]^2\right)>\frac{\text{kB} \text{p0}[\varpi ] T[\varpi
]}{\varpi }\)
\end{doublespace}

\begin{doublespace}
\noindent\(m \varpi ^2 \left(\Omega [\varpi ]^2-\text{$\Omega $k}[\varpi ]^2\right)>\text{kB} T[\varpi ]\)
\end{doublespace}

\begin{doublespace}
\noindent\(m \varpi ^2 \left(\Omega [\varpi ]^2-\text{$\Omega $k}[\varpi ]^2\right)>\frac{G m M h[\varpi ]^2}{\varpi ^3}\)
\end{doublespace}

\begin{doublespace}
\noindent\(\fbox{$\varpi ^5 \left(\Omega [\varpi ]^2-\text{$\Omega $k}[\varpi ]^2\right)>G M h[\varpi ]^2$}\)
\end{doublespace}

\subsubsection*{Because $\varpi $ $<<$ h, the RHS is much smaller than the LHS, so the pressure term is negligible. With the RHS equal to zero, we
can show that the azimuthal velocity must have a Keplarian profile}

\begin{doublespace}
\noindent\(\pmb{\text{Mz$\$$c3} = \text{Mz$\$$c2} \text{/.} h[\varpi ]\to  0 \text{/.} \text{Greater} \to  \text{Equal}}\\
\pmb{\text{Mz$\$$c3} = \text{Mz$\$$c3} \text{//} \text{Simplify}}\\
\pmb{(\text{$\Omega \$$c} = \text{Solve}[\text{Mz$\$$c3},\Omega [\varpi ]][[2,1]]) \text{//}\text{Framed}}\)
\end{doublespace}

\begin{doublespace}
\noindent\(\varpi ^5 \left(\Omega [\varpi ]^2-\text{$\Omega $k}[\varpi ]^2\right)==0\)
\end{doublespace}

\begin{doublespace}
\noindent\(\Omega [\varpi ]^2==\text{$\Omega $k}[\varpi ]^2\)
\end{doublespace}

\begin{doublespace}
\noindent\(\fbox{$\Omega [\varpi ]\to \text{$\Omega $k}[\varpi ]$}\)
\end{doublespace}

\subsection*{Determine if an inwards pressure gradient creates faster or slower rotation}

\subsubsection*{Returning to our expression before the scaling arguments were applied, set the gradient of the pressure to a constant negative value}

\begin{doublespace}
\noindent\(\pmb{\text{Mz$\$$c3} = \text{Mz$\$$c1} \text{/.} \text{Greater} \to  \text{Equal}}\\
\pmb{\text{Mz$\$$c3} = \text{Mz$\$$c3} \text{/.}D[\text{p0}[\varpi ],\varpi ]\to  -\Gamma }\\
\pmb{\text{$\$$Assumptions} = \text{$\$$Assumptions} \&\& \Gamma  > 0;}\\
\pmb{\text{Solve}[\text{Mz$\$$c3}, \Omega [\varpi ]][[2,1]] \text{//}\text{Framed}}\)
\end{doublespace}

\begin{doublespace}
\noindent\(m \varpi  \text{p0}[\varpi ] \left(\Omega [\varpi ]^2-\text{$\Omega $k}[\varpi ]^2\right)==\text{kB} T[\varpi ] \frac{\partial \text{p0}(\varpi
)}{\partial \varpi }\)
\end{doublespace}

\begin{doublespace}
\noindent\(m \varpi  \text{p0}[\varpi ] \left(\Omega [\varpi ]^2-\text{$\Omega $k}[\varpi ]^2\right)==-\text{kB} \Gamma  T[\varpi ]\)
\end{doublespace}

\begin{doublespace}
\noindent\(\fbox{$\Omega [\varpi ]\to \frac{\sqrt{-\text{kB} \Gamma  T[\varpi ]+m \varpi  \text{p0}[\varpi ] \text{$\Omega $k}[\varpi ]^2}}{\sqrt{m}
\sqrt{\varpi } \sqrt{\text{p0}[\varpi ]}}$}\)
\end{doublespace}

\subsubsection*{From the expression above, we can see that a positive value of $\Gamma $ (negative pressure gradient) \textit{ reduces }the rotation
rate}

\section*{Part d.}

\subsection*{Use the continuity equation to show that $\varpi $ \(u_{\varpi }\Sigma\) is a constant}

\subsubsection*{Insert small radial flow component into velocity expression}

\begin{doublespace}
\noindent\(\pmb{\text{u$\$$1} = \text{u$\$$0};}\\
\pmb{\text{u$\$$1}[[2,1]] = \text{u$\varpi $}[\varpi ];}\\
\pmb{\text{u$\$$1}}\)
\end{doublespace}

\begin{doublespace}
\noindent\(u[\varpi ,z]\to \{\text{u$\varpi $}[\varpi ],\varpi  \Omega [\varpi ],0\}\)
\end{doublespace}

\subsubsection*{The continuity equation is often expressed as}

\begin{doublespace}
\noindent\(\pmb{\text{C$\$$}[\rho \_,\text{u$\_$}]\text{:=} \left(\partial _t\rho \right)+ (\triangledown \cdot (\rho  u)) == 0}\)
\end{doublespace}

\subsubsection*{Now integrate the continuity equation over all space}

\begin{doublespace}
\noindent\(\pmb{\text{IC$\$$d} = \text{Integrate}[\text{C$\$$}[\rho [\varpi ,z]\text{/.}\text{$\rho \$$b1}\text{/.}\text{$\rho $0$\$$b1} ,u[\varpi
,z] \text{/.} \text{u$\$$1}][[1]] \varpi ,\{\varpi ,0,\infty \},\{\phi ,0,2\pi \},\{z,-\infty , \infty \}] == \text{C1}}\)
\end{doublespace}

\begin{doublespace}
\noindent\(\int_0^{\infty } 2 \pi  \left(\varpi  \frac{\partial \text{u$\varpi $}(\varpi )}{\partial \varpi } \Sigma [\varpi ]+\text{u$\varpi $}[\varpi
] \left(\varpi  \frac{\partial \Sigma (\varpi )}{\partial \varpi }+\Sigma [\varpi ]\right)\right) \, d\varpi ==\text{C1}\)
\end{doublespace}

\subsubsection*{By inspection, we notice that the derivative of the conserved quantity}

\begin{doublespace}
\noindent\(\pmb{\text{dC$\$$d} = \partial _{\varpi }(\varpi  \text{u$\varpi $}[\varpi ] \Sigma [\varpi ]) \text{//}\text{Simplify}}\)
\end{doublespace}

\begin{doublespace}
\noindent\(\varpi  \frac{\partial \text{u$\varpi $}(\varpi )}{\partial \varpi } \Sigma [\varpi ]+\text{u$\varpi $}[\varpi ] \left(\varpi  \frac{\partial
\Sigma (\varpi )}{\partial \varpi }+\Sigma [\varpi ]\right)\)
\end{doublespace}

\subsubsection*{Is equal to the integrand of the form of the continuity equation calculated above}

\begin{doublespace}
\noindent\(\pmb{\text{dC$\$$d} == \text{IC$\$$d}[[1,1,3]] }\)
\end{doublespace}

\begin{doublespace}
\noindent\(\text{True}\)
\end{doublespace}

\subsubsection*{So by the fundamental theorem of calculus, we may deduce that the quantity $\varpi $ \(u_{\varpi }\Sigma\) is indeed constant.}

\begin{doublespace}
\noindent\(\pmb{\text{C$\$$d} = \varpi  \text{u$\varpi $}[\varpi ] \Sigma [\varpi ] \to  \text{C1};}\)
\end{doublespace}

\subsubsection*{Unfortunately, Mathematica cannot handle symbolic integrals very effectively. A more illuminating proof would be to integrate the
continuity equation and use the divergence theorem to remove the radial integral.}

\subsection*{Compute the mass flux (accretion rate) across a given radius}

\subsubsection*{Use the conserved quantity to compute the fluid velocity}

\begin{doublespace}
\noindent\(\pmb{\text{u$\varpi \$$d} = \text{Solve}[\text{C$\$$d} \text{/.} \text{r2e},\text{u$\varpi $}[\varpi ]][[1,1]]}\)
\end{doublespace}

\begin{doublespace}
\noindent\(\text{u$\varpi $}[\varpi ]\to \frac{\text{C1}}{\varpi  \Sigma [\varpi ]}\)
\end{doublespace}

\subsubsection*{The mass flux is then the product of the density and the radial fluid velocity}

\begin{doublespace}
\noindent\(\pmb{(\text{M$\$$d} = M[\varpi ,z]\to  \text{u$\varpi $}[\varpi ] \rho [\varpi ,z] \text{/.} \text{u$\varpi \$$d}) \text{//}\text{Framed}}\)
\end{doublespace}

\begin{doublespace}
\noindent\(\fbox{$M[\varpi ,z]\to \frac{\text{C1} \rho [\varpi ,z]}{\varpi  \Sigma [\varpi ]}$}\)
\end{doublespace}

\section*{Part e.}

\subsection*{Introduce the radial flow component into the azimuthal momentum equation evaluated in the equatorial plane}

\subsubsection*{Rewrite the momentum equation, eliminating the pressure term and incorporating the viscosity}

\begin{doublespace}
\noindent\(\pmb{\text{(*}\text{M$\$$1}[\rho \_,\text{u$\_$},\text{p$\_$},\psi \_] \text{:=} \rho \left(\partial _tu\right) + \rho  \left(\left(\frac{1}{2}(\triangledown
(u.u))\right)-(u\times (\triangledown \times u))\right)==-\rho  (\triangledown  \psi ) + \mu  (\Delta  u) ;\text{*)}}\\
\pmb{\text{M$\$$1}[\rho \_,\text{u$\_$},\text{p$\_$},\psi \_] \text{:=} \rho \left(\partial _tu\right) + \rho  \left(\left(\frac{1}{2}(\triangledown
(u.u))\right)-(u\times (\triangledown \times u))\right)==-\rho  (\triangledown  \psi ) + \mu ((\triangledown (\triangledown \cdot u)) - (\triangledown
\times (\triangledown \times u)));}\)
\end{doublespace}

\subsubsection*{Evaluate the azimuthal component of the momentum equation, using information gained from previous subproblems}

\begin{doublespace}
\noindent\(\pmb{\text{Ms$\$$0} = \text{Solve}[\text{$\Omega $k$\$$0}\text{/.} \text{r2e}, M][[1,1]] ;}\\
\pmb{\text{M$\phi \$$e} = \text{M$\$$1}[\rho [\varpi ,z],u[\varpi ,z]\text{/.}\text{u$\$$1} \text{/.}\Omega [\varpi ]\to  \text{$\Omega $k}[\varpi
]\text{/.} \text{$\Omega $k$\$$0},p[\varpi ,z],\psi [\varpi ]\text{/.}\text{$\psi \$$0}\text{/.}\text{$\Omega $k$\$$0}][[\text{;;},2]] \text{//}\text{Simplify}}\\
\pmb{\text{M$\phi \$$e} = \text{M$\phi \$$e} \text{/.} \text{Ms$\$$0}\text{/.} z\to  0}\\
\pmb{\text{M$\phi \$$e} = \text{M$\phi \$$e}\text{//}\text{Simplify}}\)
\end{doublespace}

\begin{doublespace}
\noindent\(3 \mu  \sqrt{\frac{G M}{\varpi ^5}}+2 \sqrt{\frac{G M}{\varpi ^3}} \text{u$\varpi $}[\varpi ] \rho [\varpi ,z]==0\)
\end{doublespace}

\begin{doublespace}
\noindent\(2 \text{u$\varpi $}[\varpi ] \rho [\varpi ,0] \sqrt{\text{$\Omega $k}[\varpi ]^2}+3 \mu  \sqrt{\frac{\text{$\Omega $k}[\varpi ]^2}{\varpi
^2}}==0\)
\end{doublespace}

\begin{doublespace}
\noindent\(3 \mu +2 \varpi  \text{u$\varpi $}[\varpi ] \rho [\varpi ,0]==0\)
\end{doublespace}

\subsubsection*{Solve for the density in terms of the azimuthal flow velocity}

\begin{doublespace}
\noindent\(\pmb{\text{$\rho \$$e} = \text{Solve}[\text{M$\phi \$$e}, \rho [\varpi ,0]][[1,1]]}\)
\end{doublespace}

\begin{doublespace}
\noindent\(\rho [\varpi ,0]\to -\frac{3 \mu }{2 \varpi  \text{u$\varpi $}[\varpi ]}\)
\end{doublespace}

\subsubsection*{Insert this expression into our equation for the accretion rate}

\begin{doublespace}
\noindent\(\pmb{(\text{M$\$$e} = \text{M$\$$d} \text{/.} z\to  0\text{/.} \text{$\rho \$$e}\text{/.} \text{u$\varpi \$$d})\text{//}\text{Framed}}\)
\end{doublespace}

\begin{doublespace}
\noindent\(\fbox{$M[\varpi ,0]\to -\frac{3 \mu }{2 \varpi }$}\)
\end{doublespace}

\subsection*{Evaluate the accretion rate for the point on the disk used in Part a.}

\subsubsection*{Enter the viscosity of an unmagnetized, fully ionized plasma}

\begin{doublespace}
\noindent\(\pmb{\text{AppendTo}\left[\text{vals}, \mu \to  0.12 \left(T[\varpi ]\left/\left(10^6\text{Quantity}[\text{{``}KelvinsDifference{''}}]\right)\right.\right)^{5/2}\text{Quantity}\left[\frac{\text{{``}Grams{''}}}{\text{{``}Centimeters{''}}
\text{{``}Seconds{''}}} \right]\right];}\)
\end{doublespace}

\subsubsection*{Evaluate the accretion rate}

\begin{doublespace}
\noindent\(\pmb{\text{M$\$$e1} = \text{M$\$$e}}\\
\pmb{\text{M$\$$e1}[[2]] = \text{M$\$$e1}[[2]] \text{/.} \text{vals} \text{/.} \text{vals};}\)
\end{doublespace}

\begin{doublespace}
\noindent\(M[\varpi ,0]\to -\frac{3 \mu }{2 \varpi }\)
\end{doublespace}

\begin{doublespace}
\noindent\(\pmb{\text{M$\$$e1} \text{/.} \varpi \to 1 \text{Quantity}[\text{{``}AstronomicalUnit{''}}] \text{//}\text{Framed}}\)
\end{doublespace}

\begin{doublespace}
\noindent\(\fbox{$M[,0]\to $}\)
\end{doublespace}

\subsection*{Evaluate the accretion luminosity}

\subsubsection*{Save the value of the speed of light to memory}

\begin{doublespace}
\noindent\(\pmb{\text{AppendTo}[\text{vals}, c\to  \text{Quantity}[\text{{``}SpeedOfLight{''}}]];}\)
\end{doublespace}

\subsubsection*{Evaluate the quantity \(M c^2\).}

\begin{doublespace}
\noindent\(\pmb{\text{Mc$\$$e} = M[\varpi ,0]c^2\to \left(\text{  }M[\varpi ,0]c^2\text{/.} \text{M$\$$e}\text{/.} \text{vals} \text{/.} \text{vals}\right)
\text{//}\text{UnitConvert} \text{//}\text{Framed}}\)
\end{doublespace}

\begin{doublespace}
\noindent\(\fbox{$c^2 M[\varpi ,0]\to $}\)
\end{doublespace}

\subsubsection*{Compare to the solar luminosity (different units?)}

\begin{doublespace}
\noindent\(\pmb{\text{Quantity}[\text{{``}SolarLuminosity{''}}]\text{//}\text{UnitConvert}}\)
\end{doublespace}

\begin{doublespace}
\noindent\(\)
\end{doublespace}

\section*{Part f.}

\subsection*{Obtain an expression for the inward radial velocity}

\subsubsection*{We are given that the kinematic turbulent viscosity has the form}

\begin{doublespace}
\noindent\(\pmb{\text{$\nu $t$\$$f} = \nu [\varpi ]\to  \text{csi}[\varpi ] h[\varpi ];}\)
\end{doublespace}

\subsubsection*{Kinematic viscosity is related to the dynamic viscosity through}

\begin{doublespace}
\noindent\(\pmb{\text{$\mu \$$0} = \mu  \to  \nu [\varpi ]\rho [\varpi ,z];}\)
\end{doublespace}

\subsubsection*{Evaluate the accretion rate obtained in part e. with this turbulent viscosity coefficient}

\begin{doublespace}
\noindent\(\pmb{\text{M$\$$f} = \text{M$\$$e} }\\
\pmb{\text{M$\$$f} = \text{M$\$$f} \text{/.} \text{$\mu \$$0} \text{/.} z\to 0 }\\
\pmb{\text{M$\$$f} =\text{M$\$$f} \text{/.} \text{$\nu $t$\$$f}}\)
\end{doublespace}

\begin{doublespace}
\noindent\(M[\varpi ,0]\to -\frac{3 \mu }{2 \varpi }\)
\end{doublespace}

\begin{doublespace}
\noindent\(M[\varpi ,0]\to -\frac{3 \nu [\varpi ] \rho [\varpi ,0]}{2 \varpi }\)
\end{doublespace}

\begin{doublespace}
\noindent\(M[\varpi ,0]\to -\frac{3 \text{csi}[\varpi ] h[\varpi ] \rho [\varpi ,0]}{2 \varpi }\)
\end{doublespace}

\subsubsection*{Use the definition of the mass flux to solve for the radial velocity}

\begin{doublespace}
\noindent\(\pmb{(\text{u$\varpi \$$f} =\text{Solve}[\text{M$\$$f} \text{/.}\text{  }M[\varpi ,0]\to  \text{u$\varpi $}[\varpi ] \rho [\varpi ,0]
\text{/.} \text{r2e}, \text{u$\varpi $}[\varpi ]][[1,1]]) \text{//}\text{Framed}}\)
\end{doublespace}

\begin{doublespace}
\noindent\(\fbox{$\text{u$\varpi $}[\varpi ]\to -\frac{3 \text{csi}[\varpi ] h[\varpi ]}{2 \varpi }$}\)
\end{doublespace}

\subsection*{Show for a thin disk that \(\left\|u_{\varpi }\right.\)$||$ $<<$ \(u_{\phi }\)}

\subsubsection*{We can relate the azimuthal velocity to the sound speed through the vertical scale height}

\begin{doublespace}
\noindent\(\pmb{\text{$\Omega $k$\$$f} = \text{Solve}[\text{h$\$$a} \text{/.} \text{r2e}, \text{$\Omega $k}[\varpi ]][[1,1]]}\)
\end{doublespace}

\begin{doublespace}
\noindent\(\text{$\Omega $k}[\varpi ]\to \frac{\text{csi}[\varpi ]}{h[\varpi ]}\)
\end{doublespace}

\subsubsection*{Plugging this expression into the azimuthal velocity yields}

\begin{doublespace}
\noindent\(\pmb{\text{u$\phi \$$f} =\text{u$\phi $}[\varpi ] \to  \text{u$\$$1}[[2,2]] \text{/.} \Omega  \to  \text{$\Omega $k} \text{/.} \text{$\Omega
$k$\$$f}}\)
\end{doublespace}

\begin{doublespace}
\noindent\(\text{u$\phi $}[\varpi ]\to \frac{\varpi  \text{csi}[\varpi ]}{h[\varpi ]}\)
\end{doublespace}

\subsubsection*{Comparing the above azimuthal velocity to the radial velocity shows}

\begin{doublespace}
\noindent\(\pmb{\text{Abs}[(\text{u$\varpi $}[\varpi ]\text{/.}\text{u$\varpi \$$f})] < (\text{u$\phi $}[\varpi ] \text{/.} \text{u$\phi \$$f})\text{
 }\text{//}\text{Simplify} \text{//}\text{Framed}}\)
\end{doublespace}

\begin{doublespace}
\noindent\(\fbox{$3 h[\varpi ]^2<2 \varpi ^2$}\)
\end{doublespace}

\subsubsection*{Since h $<<$ $\varpi $, the azimuthal flow velocity is indeed much larger than the radial velocity}

\subsection*{Show for a thin disk that \(\left\|u_{\varpi }\right.\)$||$ $<<$ \(c_{s,i}\)}

\subsubsection*{Comparing the radial flow velocity to the sound speed gives the inequality}

\begin{doublespace}
\noindent\(\pmb{(\text{Abs}[(\text{u$\varpi $}[\varpi ]\text{/.}\text{u$\varpi \$$f})] < \text{csi}[\varpi ] ) \text{//}\text{Simplify} \text{//}\text{Framed}}\)
\end{doublespace}

\begin{doublespace}
\noindent\(\fbox{$3 h[\varpi ]<2 \varpi $}\)
\end{doublespace}

\subsubsection*{Since h $<<$ $\varpi $, the sound speed is also much larger than the radial velocity}

\section*{Part g.}

\subsection*{The problem statement instructs us to consider an isothermal sound speed}

\begin{doublespace}
\noindent\(\pmb{\text{csi$\$$g} = \text{csi}[\varpi ] \to  \xi \sqrt{\frac{G M}{\varpi }};}\)
\end{doublespace}

\subsection*{Find the requirements of $\xi $ for a thin disk}

\subsubsection*{Write the above isothermal sound speed in terms of the Keplarian rotation}

\begin{doublespace}
\noindent\(\pmb{\text{csi$\$$g1}= \text{csi$\$$g}\text{/.} \text{Solve}[\text{$\Omega $k$\$$0}\text{/.} \text{r2e}, M][[1,1]] \text{//}\text{Simplify}}\)
\end{doublespace}

\begin{doublespace}
\noindent\(\text{csi}[\varpi ]\to \xi  \varpi  \text{$\Omega $k}[\varpi ]\)
\end{doublespace}

\subsubsection*{For a thin disk, h $<<$ $\varpi $, using the definition of h[$\varpi $] defined in Part a., we have,}

\begin{doublespace}
\noindent\(\pmb{\text{e$\$$g1} =(h[\varpi ] ) < \varpi  }\\
\pmb{\text{e$\$$g2} = \text{e$\$$g1} \text{/.} \text{h$\$$a} \text{/.} \text{csi$\$$g1}}\\
\pmb{\text{e$\$$g2} \text{//}\text{FullSimplify} \text{//}\text{Framed}}\)
\end{doublespace}

\begin{doublespace}
\noindent\(h[\varpi ]<\varpi\)
\end{doublespace}

\begin{doublespace}
\noindent\(\xi  \varpi <\varpi\)
\end{doublespace}

\begin{doublespace}
\noindent\(\fbox{$\xi <1$}\)
\end{doublespace}

\subsubsection*{So the constant $\xi $ must be much smaller than one and positive for the thin disk limit to hold.}

\begin{doublespace}
\noindent\(\pmb{\text{$\$$Assumptions} = \text{$\$$Assumptions} \&\& \xi  >0 \&\& \xi  < 1;}\)
\end{doublespace}

\subsection*{Compute the trajectory of the fluid element}

\subsubsection*{The fluid element{'}s trajectory satisfies the equation}

\begin{doublespace}
\noindent\(\pmb{\text{T$\$$g} = \partial _t\{\varpi [t], \phi [t],z[t]\} == (u[\varpi ,z])}\\
\pmb{\text{T$\$$g} = \text{T$\$$g} \text{/.} \text{u$\$$1} \text{/.} \Omega \to  \text{$\Omega $k}}\\
\pmb{\text{T$\$$g} = \text{T$\$$g}\text{   }\text{/.} \text{u$\varpi \$$f}}\\
\pmb{\text{T$\$$g} = \text{T$\$$g} \text{/.} \text{h$\$$a} }\\
\pmb{\text{T$\$$g} = \text{T$\$$g} \text{/.} \text{csi$\$$g1}}\\
\pmb{\text{T$\$$g}[[2]] = (\text{T$\$$g}[[2]] \text{/.} \text{$\Omega $k$\$$0} \text{//}\text{Simplify})\text{/.} \varpi \to  \varpi [t];}\\
\pmb{\text{T$\$$g}}\)
\end{doublespace}

\begin{doublespace}
\noindent\(\left\{\frac{\partial \varpi (t)}{\partial t},\frac{\partial \phi (t)}{\partial t},\frac{\partial z(t)}{\partial t}\right\}==u[\varpi
,z]\)
\end{doublespace}

\begin{doublespace}
\noindent\(\left\{\frac{\partial \varpi (t)}{\partial t},\frac{\partial \phi (t)}{\partial t},\frac{\partial z(t)}{\partial t}\right\}==\{\text{u$\varpi
$}[\varpi ],\varpi  \text{$\Omega $k}[\varpi ],0\}\)
\end{doublespace}

\begin{doublespace}
\noindent\(\left\{\frac{\partial \varpi (t)}{\partial t},\frac{\partial \phi (t)}{\partial t},\frac{\partial z(t)}{\partial t}\right\}==\left\{-\frac{3
\text{csi}[\varpi ] h[\varpi ]}{2 \varpi },\varpi  \text{$\Omega $k}[\varpi ],0\right\}\)
\end{doublespace}

\begin{doublespace}
\noindent\(\left\{\frac{\partial \varpi (t)}{\partial t},\frac{\partial \phi (t)}{\partial t},\frac{\partial z(t)}{\partial t}\right\}==\left\{-\frac{3
\text{csi}[\varpi ]^2}{2 \varpi  \text{$\Omega $k}[\varpi ]},\varpi  \text{$\Omega $k}[\varpi ],0\right\}\)
\end{doublespace}

\begin{doublespace}
\noindent\(\left\{\frac{\partial \varpi (t)}{\partial t},\frac{\partial \phi (t)}{\partial t},\frac{\partial z(t)}{\partial t}\right\}==\left\{-\frac{3}{2}
\xi ^2 \varpi  \text{$\Omega $k}[\varpi ],\varpi  \text{$\Omega $k}[\varpi ],0\right\}\)
\end{doublespace}

\begin{doublespace}
\noindent\(\left\{\frac{\partial \varpi (t)}{\partial t},\frac{\partial \phi (t)}{\partial t},\frac{\partial z(t)}{\partial t}\right\}==\left\{-\frac{3}{2}
\xi ^2 \sqrt{\frac{G M}{\varpi [t]}},\sqrt{\frac{G M}{\varpi [t]}},0\right\}\)
\end{doublespace}

\subsubsection*{Solve the above set of differential equations for the trajectory}

\begin{doublespace}
\noindent\(\pmb{\text{$\$$Assumptions} = \text{$\$$Assumptions} \&\& R >0 \&\& t>0;}\\
\pmb{\{\text{$\varpi \$$g},\text{$\phi \$$g},\text{z$\$$g}\}=\text{DSolve}[\{\text{T$\$$g}, \varpi [0] == R, \phi [0] == 0, z[0] ==0\},\{\varpi [t],
\phi [t],z[t]\},t][[2]] \text{//}\text{Simplify}}\)
\end{doublespace}

\begin{doublespace}
\noindent\(\left\{\varpi [t]\to \frac{\left(-4 R^{3/2}+9 \sqrt{G M} t \xi ^2\right)^{2/3}}{2\ 2^{1/3}},\phi [t]\to \frac{-4 R+(G M)^{1/3} \left(-\frac{8
R^{3/2}}{\sqrt{G M}}+18 t \xi ^2\right)^{2/3}}{6 \xi ^2},z[t]\to 0\right\}\)
\end{doublespace}

\subsubsection*{Solve for t in terms of the radial coordinate}

\begin{doublespace}
\noindent\(\pmb{\text{t$\varpi \$$g} = \text{Solve}[\text{$\varpi \$$g} \text{/.} \varpi [t]\to  \varpi \text{/.} \text{r2e}, t][[1,1]] \text{//}\text{Simplify}}\)
\end{doublespace}

\begin{doublespace}
\noindent\(t\to \frac{4 \left(R^{3/2}+\varpi ^{3/2}\right)}{9 \sqrt{G M} \xi ^2}\)
\end{doublespace}

\subsubsection*{Solve for t in terms of the azimuthal coordinate}

\begin{doublespace}
\noindent\(\pmb{\text{t$\phi \$$g}=\text{Solve}[\text{$\phi \$$g} \text{/.} \phi [t]\to  \phi \text{/.} \text{r2e}, t][[1,1]] \text{//}\text{Simplify}}\)
\end{doublespace}

\begin{doublespace}
\noindent\(t\to \frac{4 R^{3/2}+2 R \sqrt{4 R+6 \xi ^2 \phi }+3 \xi ^2 \phi  \sqrt{4 R+6 \xi ^2 \phi }}{9 \sqrt{G M} \xi ^2}\)
\end{doublespace}

\subsubsection*{Equate both expressions for t to find the trajectory in the form $\varpi $($\phi $)}

\begin{doublespace}
\noindent\(\pmb{\text{$\$$Assumptions} = \text{$\$$Assumptions} \&\& \phi  >0\&\& R > 0;}\\
\pmb{(\text{$\varpi \phi \$$g} =\text{Solve}[(t \text{/.}\text{t$\varpi \$$g})==(t \text{/.}\text{t$\phi \$$g}),\varpi ][[1,1]] \text{//}\text{FullSimplify})
\text{//}\text{Framed}}\)
\end{doublespace}

\begin{doublespace}
\noindent\(\fbox{$\varpi \to R+\frac{3 \xi ^2 \phi }{2}$}\)
\end{doublespace}

\subsection*{Compute the time required for a fluid element to fall onto the central mass}

\subsubsection*{Use the expression for t[$\varpi $] calculated in the previous section of the problem to calculate the fall time by setting $\varpi
$=0.}

\begin{doublespace}
\noindent\(\pmb{\text{Tf$\$$g} = \text{Tf}\to  (t\text{/.}\text{t$\varpi \$$g} \text{/.} \varpi \to  0)}\\
\pmb{\text{Tf$\$$g} = \text{Tf$\$$g} \text{/.} \text{Solve}[\text{$\Omega $k$\$$0} \text{/.} \varpi \to  R\text{/.} \text{r2e}, M][[1,1]] }\\
\pmb{\text{$\$$Assumptions} = \text{$\$$Assumptions} \&\& \text{$\Omega $k}[R] > 0;}\\
\pmb{\text{Tf$\$$g} = \text{Tf$\$$g}\text{  }\text{//}\text{Simplify}}\\
\pmb{(\text{Tf$\$$g} = \text{Tf$\$$g} \text{/.} \text{$\Omega $k}[R]\to  2 \pi /\text{Tk}[R]) \text{//}\text{Framed}}\)
\end{doublespace}

\begin{doublespace}
\noindent\(\text{Tf}\to \frac{4 R^{3/2}}{9 \sqrt{G M} \xi ^2}\)
\end{doublespace}

\begin{doublespace}
\noindent\(\text{Tf}\to \frac{4 R^{3/2}}{9 \xi ^2 \sqrt{R^3 \text{$\Omega $k}[R]^2}}\)
\end{doublespace}

\begin{doublespace}
\noindent\(\text{Tf}\to \frac{4}{9 \xi ^2 \text{$\Omega $k}[R]}\)
\end{doublespace}

\begin{doublespace}
\noindent\(\fbox{$\text{Tf}\to \frac{2 \text{Tk}[R]}{9 \pi  \xi ^2}$}\)
\end{doublespace}

\section*{Part h.}

\subsection*{Solve for acoustic wave modes by using the eikenol approximation with the phase function}

\begin{doublespace}
\noindent\(\pmb{\text{$\varphi \$$h} = \varphi [\varpi ,\phi ]\to  m(f[\varpi ]-\phi );}\)
\end{doublespace}

\subsection*{This phase function should satisfy the dispersion relation}

\begin{doublespace}
\noindent\(\pmb{\text{$\omega \$$h}[\text{u$\_$},\varphi \_] \text{:=}\omega  ==\text{  }u.(\triangledown  \varphi ) +\text{csa}[\varpi ]\text{Norm}[(\triangledown
 \varphi )] }\)
\end{doublespace}

\subsection*{Where the adiabatic sound speed is given by}

\begin{doublespace}
\noindent\(\pmb{\text{csa$\$$h} = \text{csa}[\varpi ]\to  \sqrt{\gamma }\text{csi}[\varpi ];}\)
\end{doublespace}

\begin{doublespace}
\noindent\(\pmb{\text{$\$$Assumptions} = \text{$\$$Assumptions} \&\& \gamma  > 0;}\)
\end{doublespace}

\subsection*{Solve for the unknown function f[$\varpi $] for standing wave solutions ($\omega $ = 0)}

\begin{doublespace}
\noindent\(\pmb{\text{f$\$$h} = \text{$\omega \$$h}[u[\varpi ,z]\text{/.} \text{u$\$$0}, \varphi [\varpi ,\phi ] \text{/.}\text{$\varphi \$$h} ]}\\
\pmb{\text{f$\$$h} = \text{f$\$$h}\text{/.} \text{csa$\$$h}\text{  }\text{/.} \omega \to  0}\\
\pmb{\text{f$\$$h} = \text{f$\$$h} \text{/.} \text{csi$\$$g}}\\
\pmb{\text{f$\$$h} = \text{f$\$$h} \text{/.} \Omega \to  \text{$\Omega $k} \text{/.} \text{$\Omega $k$\$$0} \text{//}\text{Simplify}}\\
\pmb{(\text{f$\$$h} = \text{DSolve}[\text{f$\$$h},f[\varpi ],\varpi ][[1,1]]) \text{/.} C[1]\to  0 \text{//}\text{Framed} \text{//}\text{Quiet}}\)
\end{doublespace}

\begin{doublespace}
\noindent\(\omega ==\sqrt{\text{Abs}\left[\frac{m}{\varpi }\right]^2+\text{Abs}\left[m \frac{\partial f(\varpi )}{\partial \varpi }\right]^2} \text{csa}[\varpi
]-m \Omega [\varpi ]\)
\end{doublespace}

\begin{doublespace}
\noindent\(0==\sqrt{\gamma } \sqrt{\text{Abs}\left[\frac{m}{\varpi }\right]^2+\text{Abs}\left[m \frac{\partial f(\varpi )}{\partial \varpi }\right]^2}
\text{csi}[\varpi ]-m \Omega [\varpi ]\)
\end{doublespace}

\begin{doublespace}
\noindent\(0==\sqrt{\gamma } \xi  \sqrt{\frac{G M}{\varpi }} \sqrt{\text{Abs}\left[\frac{m}{\varpi }\right]^2+\text{Abs}\left[m \frac{\partial f(\varpi
)}{\partial \varpi }\right]^2}-m \Omega [\varpi ]\)
\end{doublespace}

\begin{doublespace}
\noindent\(\sqrt{\frac{G M}{\varpi ^3}}==\xi  \sqrt{\frac{G M \gamma  \left(1+\varpi ^2 \text{Abs}\left[\frac{\partial f(\varpi )}{\partial \varpi
}\right]^2\right)}{\varpi ^3}}\)
\end{doublespace}

\begin{doublespace}
\noindent\(\fbox{$f[\varpi ]\to -\frac{\sqrt{\frac{1}{\gamma }-\xi ^2} \text{Log}[\varpi ]}{\xi }$}\)
\end{doublespace}

\subsection*{Show that density ridges follow a spiral}

\subsubsection*{Plug the f[$\varpi $] found in the previous step into the phase function}

\begin{doublespace}
\noindent\(\pmb{\text{Solve}[\{\text{$\varphi \$$h}\text{/.} \text{f$\$$h} \text{/.} \text{r2e} \text{/.} \varphi [\varpi , \phi ]\to  0 \},\varpi
][[1,1]]}\)
\end{doublespace}

\begin{doublespace}
\noindent\(\varpi \to e^{-\frac{\sqrt{\gamma } \xi  \phi }{\sqrt{1-\gamma  \xi ^2}}+\frac{\sqrt{\gamma } \xi  C[1]}{\sqrt{1-\gamma  \xi ^2}}}\)
\end{doublespace}

\subsubsection*{This spiral is much tighter than the accretion trajectory since it grows exponentially rather than linearly.}

\end{document}
\grid
